\documentclass{article}

\usepackage[utf8]{inputenc}
\usepackage[russian]{babel}
\usepackage[a4paper, margin=1in]{geometry}
\usepackage{graphicx}
\usepackage{amsmath}
\usepackage{wrapfig}
\usepackage{multirow}
\usepackage{mathtools}
\usepackage{pgfplots}
\usepackage{pgfplotstable}
\usepackage{setspace}
\usepackage{changepage}
\usepackage{caption}
\usepackage{csquotes}
\usepackage{hyperref}
\usepackage{listings}

\pgfplotsset{compat=1.18}
\hypersetup{
  colorlinks = true,
  linkcolor  = blue,
  filecolor  = magenta,      
  urlcolor   = darkgray,
  pdftitle   = {
    ai-ontology-report-smirnov-victor-p33131
  },
}

\definecolor{codegreen}{rgb}{0,0.6,0}
\definecolor{codegray}{rgb}{0.5,0.5,0.5}
\definecolor{codepurple}{rgb}{0.58,0,0.82}
\definecolor{backcolour}{rgb}{0.99,0.99,0.99}
\lstdefinestyle{codestyle}{
  backgroundcolor=\color{backcolour},   
  commentstyle=\color{codegreen},
  keywordstyle=\color{magenta},
  numberstyle=\tiny\color{codegray},
  stringstyle=\color{codepurple},
  basicstyle=\ttfamily\footnotesize,
  breakatwhitespace=false,         
  breaklines=true,                 
  captionpos=b,                    
  keepspaces=true,                 
  numbers=left,                    
  numbersep=5pt,                  
  showspaces=false,                
  showstringspaces=false,
  showtabs=false,                  
  tabsize=2
}


\lstset{style=codestyle}

\begin{document}

\begin{titlepage}
  \begin{center}
    \begin{spacing}{1.4}
      \large{Университет ИТМО} \\
      \large{Факультет программной инженерии и компьютерной техники} \\
    \end{spacing}
    \vfill
    \textbf{
      \huge{Системы искусственного интеллекта.} \\
      \huge{Лабораторные работы 1-3.} \\
      \huge{Базы знаний и онтологии} \\
    }
  \end{center}
  \vfill
  \begin{center}
    \begin{tabular}{r l}
      Группа : & P33131                  \\
      Студент: & Смирнов Виктор Игоревич \\
      Тема   : & Игра Terraria           \\
    \end{tabular}
  \end{center}
  \vfill
  \begin{center}
    \begin{large}
      2023
    \end{large}
  \end{center}
\end{titlepage}

\section*{Ключевые слова}
База знаний, Онтология.

\tableofcontents

\section{Введение}

\subsection{Цели}

Целью лабораторных работ первого модуля
была разработка базы знаний на базе языка
Prolog, а также фронтэнда к ней на Python для
взаимодейстия пользователя с системой.
Прежде всего это был просто образовательный
проект с целью ознакомления с онтологиями.

\subsection{Значимость}

Значимость заключается в полученных мною знаний
и создании автоматиеского чат-бота для получения
персональных рекомендаций по игре Terraria.

\section{Анализ требований}

\subsection{Требования к фронтэнду}

\begin{enumerate}
  \item Фронтэнд должен обрабатывать запросы пользователя и выдавать
        на них ответы, соответвующие состоянию базы знаний.
  \item Фронтэнд должен быть устойчив к неккоректному
        пользовательскому вводу.
  \item Фронтэнд должен быть реализован на языке Python.
  \item Фронтэнд должен иметь CLI интерфейс.
  \item Фронтэнд должен инициировать взаимодейстие с базой знаний.
\end{enumerate}

\subsection{Требования к базе знаний}

\begin{enumerate}
  \item База зананий должна быть реализована на базе системы Prolog.
  \item База знаний должна содержать информацию об игре Terraria.
  \item База знаний должна содержать факты и правила различной сложности.
\end{enumerate}


\section{Изучение основных концепций и инструментов}

\subsection{Обзор основных концепций баз знаний и онтологий}

По определению из википедии база знаний - база данных, содержащая
правила вывода и информацию о знаниях в некоторой предметной
области.

Онтология -- иерархический способ представления в базе знаний
набора понятий и их связей.

Важно, что онтология сама по себе содержит только абстрактные
понятия и связи. Она становится базой знаний, только когда там
появляются конкретные объекты.

Базы знаний способны делать автоматические умозаключения об уже
имеющихся или вновь вводимых фактах и тем самым производить
как бы осмысленную обработку информации.

База знаний содержит правила вывода -- логические конструкции,
определяющие отношения между фактами и позволяющие делать выводы.

База знаний обрабатывает запросы к себе, выдавая информацию,
удовлетворяющую заданным в запросе условиям.

Онтологии состоят из
\begin{enumerate}
  \item экземпляров - конкретных или абстрактных объектов,
  \item понятий - абстрактных наборов объектов,
  \item атрибутов - некоторых значений, связанных с конкретным экземпляром,
  \item отношений - завсимостей между объектами онтологии.
\end{enumerate}

Онтология позволяет формализовать факты и правила и обрабатывать их
автоматически, например, в системах искусственного интеллекта или
для анализа данных.

Онтология задает смысл данных, а база знаний их хранит и предоставляет
доступ.

\subsection{Prolog и его возможностей для разработки систем искусственного интеллекта}

Prolog (Programming in Logic) - это декларативный язык
программирования, который используется для решения задач
искусственного интеллекта и логического программирования.
Prolog основан на логике первого порядка и предназначен для
решения задач, в которых логические отношения и правила
играют ключевую роль.

\subsection{Инструменты и библиотеки, подходящие для работы с базами знаний и онтологиями на Prolog}

.

Инструмент для Базы знаний -
\url{https://swish.swi-prolog.org/}.

Инструмент для Онтологий -
\url{https://protege.stanford.edu/}.

Инструмент для взаимодейстия с базой знаний -
\url{https://github.com/yuce/pyswip/}.


\section{Создание правил и логики вывода для принятия решений на основе базы знаний и онтологии}

\subsection{Создание правил и логики вывода для принятия решений на основе базы знаний и онтологии}

Была составлена база знаний на языке Prolog.

\lstinputlisting[
  language={Prolog},
  caption={Достижения игрока}
]{../res/achievements.pl}

\lstinputlisting[
  language={Prolog},
  caption={Доступность предмета игроку в один шаг}
]{../res/available.pl}

\lstinputlisting[
  language={Prolog},
  caption={Доступность предмета игроку в несколько шагов}
]{../res/reachable.pl}

\lstinputlisting[
  language={Prolog},
  caption={Злые биомы Террарии}
]{../res/evil.pl}

\lstinputlisting[
  language={Prolog},
  caption={Руды Террарии}
]{../res/ore.pl}

\lstinputlisting[
  language={Prolog},
  caption={Кирки Террарии}
]{../res/pickaxe.pl}

\lstinputlisting[
  language={Prolog},
  caption={Прогресс игрока}
]{../res/progress.pl}

\lstinputlisting[
  language={Prolog},
  caption={Требования для добычи предметов}
]{../res/requirement.pl}

\subsection{Онтология в Protege}

\url{https://github.com/vityaman-edu/ai-prolog/blob/trunk/res/ai-lab-2.rdf}

\subsection{Тестирование и отладка системы, обеспечение ее функциональности и эффективности}

Была написана система принятия решений в виде CLI чат-бота на
языке Python.

Запросы к базе зананий выполнялись с помощью библиотеки pyswip.

\url{https://github.com/vityaman-edu/ai-prolog/blob/trunk/src/prolog.py}

Также в Python была выражена часть предметной области для
функции для построения запросов к серверу Prolog и
удобного тестирования.

\url{https://github.com/vityaman-edu/ai-prolog/blob/trunk/src/dsl.py}

Для разбора ответов сервера Prolog был сделан простенький
парсер выражений на основе Python Lark.

\url{https://github.com/vityaman-edu/ai-prolog/blob/trunk/src/parse.py}

На базе описанного выше удалось спрятать Prolog за слоем
бизнес логики.

\url{https://github.com/vityaman-edu/ai-prolog/blob/trunk/src/terraria.py}

Слой бизнес логики используется слоем принятия и обработки
текстовых запросов.

\url{https://github.com/vityaman-edu/ai-prolog/blob/trunk/src/feedback.py}

Ну и вишенкой на торте стало CLI приложение.

\url{https://github.com/vityaman-edu/ai-prolog/blob/trunk/src/args.py}

\url{https://github.com/vityaman-edu/ai-prolog/blob/trunk/src/main.py}

Програмнное обеспечение всегда страшно разрабатывать без тестов,
поэтому без них не обошлось.

Было проведено модульное тестирование парсера.

\url{https://github.com/vityaman-edu/ai-prolog/blob/trunk/src/test_parse.py}

Также слегка проверен был слой бизнес-логики.

\url{https://github.com/vityaman-edu/ai-prolog/blob/trunk/src/test_terraria.py}

Чтобы быть более уверенным в корректной обработки
даже неправильных запросов пользователя была реализована
некоторая пародия на фаззинг-тестирование.

\url{https://github.com/vityaman-edu/ai-prolog/blob/trunk/src/test_fuzzing.py}

\subsection{Пример вывода программы}

\url{https://github.com/vityaman-edu/ai-prolog/blob/trunk/res/example.txt}


\section{Оценка и интерпретация результатов}

Наша система соответвуюет заявленным требованиям.

Цель была достигнута - мы разработали CLI чат-бота предоставляющего
доступ к базе знаний на языке Prolog, он помогает человеку
играть в игру Terraria, советуя, какие шаги предпринять и
давая комментарии по поводу прогресса в игре.

В дальнейшем можно расширять базу знаний, добавлять новые аспекты
игры. Расширить вариативность как ввода, так и вывода. Сделать
бота более гибким и похожим на человека. Также следует исправить
существующие проблемы приложения (баги) и повысить тестовое покрытие,
доработать фаззинг.

\section{Заключение}

Разработанное приложение поможет игрокам в Terraria понять
правила игры и получить подсказки.

\begin{thebibliography}{9}

  \bibitem{ontology}
  \url{https://ru.wikipedia.org/wiki/%D0%9E%D0%BD%D1%82%D0%BE%D0%BB%D0%BE%D0%B3%D0%B8%D1%8F_(%D0%B8%D0%BD%D1%84%D0%BE%D1%80%D0%BC%D0%B0%D1%82%D0%B8%D0%BA%D0%B0)}

  \bibitem{knowledge}
  \url{https://ru.wikipedia.org/wiki/%D0%91%D0%B0%D0%B7%D0%B0_%D0%B7%D0%BD%D0%B0%D0%BD%D0%B8%D0%B9}

  \bibitem{prolog}
  \url{https://www.swi-prolog.org/}

  \bibitem{protege}
  \url{https://protege.stanford.edu/}

  \bibitem{pyswip}
  \url{https://github.com/yuce/pyswip}

\end{thebibliography}

\end{document}